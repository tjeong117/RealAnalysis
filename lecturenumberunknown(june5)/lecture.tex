\documentclass{article}
\usepackage{amsmath, amsthm, amssymb, amsfonts}
\usepackage{thmtools}
\usepackage{graphicx}
\usepackage{setspace}
\usepackage{geometry}
\usepackage{float}
\usepackage{hyperref}
\usepackage[utf8]{inputenc}
\usepackage[english]{babel}
\usepackage{framed}
\usepackage[dvipsnames]{xcolor}
\usepackage{MnSymbol} 
\usepackage{tcolorbox}

\colorlet{LightGray}{White!90!Periwinkle}
\colorlet{LightOrange}{Orange!15}
\colorlet{LightGreen}{Green!15}

\newcommand{\HRule}[1]{\rule{\linewidth}{#1}}

\colorlet{LightGray}{black!10}
\colorlet{LightOrange}{orange!15}
\colorlet{LightGreen}{green!15}
\colorlet{LightBlue}{blue!15}
\colorlet{LightCyan}{cyan!15}
\colorlet{whiiii}{white!15}



\declaretheoremstyle[name=Theorem,]{thmsty}
\declaretheorem[style=thmsty,numberwithin=section]{theorem}
\usepackage{tcolorbox} % Add missing package
\tcolorboxenvironment{theorem}{colback=LightGray}


\declaretheoremstyle[name=Principle,]{thmsty}
\declaretheorem[style=thmsty,numberwithin=section]{principle}
\usepackage{tcolorbox} % Add missing package
\tcolorboxenvironment{principle}{colback=LightGray}

\declaretheoremstyle[name=Definition,]{thmsty}
\declaretheorem[style=thmsty,numberwithin=section]{definition}
\tcolorboxenvironment{definition}{colback=LightBlue}

\declaretheoremstyle[name=Proposition,]{prosty}
\declaretheorem[style=prosty,numberlike=theorem]{proposition}
\tcolorboxenvironment{proposition}{colback=LightOrange}


\declaretheoremstyle[name=Proof,]{prosty}
\declaretheorem[style=prosty,numberlike=theorem]{proofbox}
\tcolorboxenvironment{proofbox}{colback=whiiii}

\declaretheoremstyle[name=Axiom,]{prcpsty}
\declaretheorem[style=prcpsty,numberlike=theorem]{axiom}
\tcolorboxenvironment{axiom}{colback=LightGreen}

\declaretheoremstyle[name=Lemma,]{prcpsty}
\declaretheorem[style=prcpsty,numberlike=theorem]{lemma}
\tcolorboxenvironment{lemma}{colback=LightCyan}





\setstretch{1.2}
\geometry{
    textheight=9in,
    textwidth=5.5in,
    top=1in,
    headheight=12pt,
    headsep=25pt,
    footskip=30pt
}

% ------------------------------------------------------------------------------

\begin{document}

% ------------------------------------------------------------------------------
% Cover Page and ToC
% ------------------------------------------------------------------------------

\title{ \normalsize \textsc{}
		\\ [2.0cm]
		\HRule{1.5pt} \\
		\LARGE \textbf{\uppercase{Lecture x (June 5)}
		\HRule{2.0pt} \\ [0.6cm] \LARGE{Comparison Theorem, Triangle Inequalities, Bolzao Weirstrass Thoerem } \vspace*{10\baselineskip}}
		}
\date{}
\author{\textbf{Author} \\ 
		Tom Jeong
        }

\maketitle
\newpage

\tableofcontents
\newpage

% ------------------------------------------------------------------------------
\section{q2}
question: $a_{n+1} = \sin(a_n)$, $a_1 = \frac{1}{100}$
a. show $(a_n)$ is converging (decreasing)
b. show $a_n \rightarrow 0$ u can use taylor series 

\begin{proofbox}
    \begin{align*}
        b_{n+1} &= b_n - \cfrac{b_n^3}{6} + b_n^4 \\
        sin(x) &= x - \frac{x^3}{6} + \frac{\sin^{(4)}(c)}{4!} \text{;; taylor series with remainder and $ 0 \leq c \leq x$} \\
    \end{align*}
    Now show $a_n \leq b_n$ (squeeze theorem) \\ 
    \begin{align*}
        a_1 &= b_1 \\ 
        a_n &\leq b_n \text{;; induction assume this and show } \\
        a_{n+1} &\leq b_{n+1}\\ 
        \sin(a_n) &\leq b_n - \cfrac{b_n^3}{6} + b_n^4 \\
        \sin(a_n) &= a_n - \cfrac{a_n^3}{6} + \cfrac{\sin(a_n)}{4!}a_n^4 \leq b_n - \cfrac{b_n^3}{6} + b_n^4 \\
        (b_n - a_n) &- \cfrac{b_n^3}{6}  + \cfrac{a_n^3}{6} + b_n^4 - a_n^4 \geq 0? \\
    \end{align*}

    let $ l = \lim a_n$ show that $l$ satisfies $l = \sin(l)$
    $$|l - \sin(l)| = |l - a_{n+1} +a_{n+1} - \sin{l} |\leq |a_{n+1} -l| + |\sin(a_n) -\sin(l)| \leq \epsilon$$ 
    
\end{proofbox}

\section{Q4}
every convergence sequence goes to 1. but $(a_n) $ does not converge. is it possible for $(a_n)$ to be bounded?
tge answer is no 
$(a_n)$ will have limsup, lim inf 
but $a_n$ doesn't converge so limsup $\neq$ lim inf (for not convergence part) 
for goeing to 1 $liminf \leq 1 \leq limsup$ then we have convergence seequence $a > 1$ which is a contradiction 


\section{Q5}
$(a_n)$ and $C = \{\text{cluster pts of } (a_n)\}$
$(b_m)$ such that $b_m \in C$
$\forall M, \exists (n_k) \text{ such that } a_{n_k} \rightarrow b_m$
b is a cluster pt of $b_m$ meaning b is a cluster pt of $(a_n)$ 

$\exists (b_{m_j}) \leq (b_m) b_{m_j}  \rightarrow b$ \\
let $\epsilon > 0$ $\exists J$ s.t. $\forall j \geq J$ $|b_{m_j} - b| < \epsilon$
$$\exists K \text{ st } \forall k \geq K \exists a_{n_k}(m_j) \rightarrow b_{m_j}$$ 
$$|a_{n_k (m_j)} - b_{m_j} | < \epsilon$$ 
$$|a_{n_k(m_j)} - b| \leq |a_{n_k(m_j)} - b_{m_j}| + |b_{m_j} - b| < 2\epsilon$$




\end{document}


