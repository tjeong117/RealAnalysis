\documentclass{article}
\usepackage{amsmath, amsthm, amssymb, amsfonts}
\usepackage{thmtools}
\usepackage{graphicx}
\usepackage{setspace}
\usepackage{geometry}
\usepackage{float}
\usepackage{hyperref}
\usepackage[utf8]{inputenc}
\usepackage[english]{babel}
\usepackage{framed}
\usepackage[dvipsnames]{xcolor}
\usepackage{tcolorbox}

\colorlet{LightGray}{White!90!Periwinkle}
\colorlet{LightOrange}{Orange!15}
\colorlet{LightGreen}{Green!15}

\newcommand{\HRule}[1]{\rule{\linewidth}{#1}}

\colorlet{LightGray}{black!10}
\colorlet{LightOrange}{orange!15}
\colorlet{LightGreen}{green!15}
\colorlet{LightBlue}{blue!15}
\colorlet{LightCyan}{cyan!15}



\declaretheoremstyle[name=Theorem,]{thmsty}
\declaretheorem[style=thmsty,numberwithin=section]{theorem}
\usepackage{tcolorbox} % Add missing package
\tcolorboxenvironment{theorem}{colback=LightGray}

\declaretheoremstyle[name=Definition,]{thmsty}
\declaretheorem[style=thmsty,numberwithin=section]{definition}
\tcolorboxenvironment{definition}{colback=LightBlue}

\declaretheoremstyle[name=Proposition,]{prosty}
\declaretheorem[style=prosty,numberlike=theorem]{proposition}
\tcolorboxenvironment{proposition}{colback=LightOrange}


\declaretheoremstyle[name=Proof,]{prosty}
\declaretheorem[style=prosty,numberlike=theorem]{proofbox}
\tcolorboxenvironment{proofbox}{colback=LightOrange}

\declaretheoremstyle[name=Axiom,]{prcpsty}
\declaretheorem[style=prcpsty,numberlike=theorem]{axiom}
\tcolorboxenvironment{axiom}{colback=LightGreen}

\declaretheoremstyle[name=Lemma,]{prcpsty}
\declaretheorem[style=prcpsty,numberlike=theorem]{lemma}
\tcolorboxenvironment{lemma}{colback=LightCyan}





\setstretch{1.2}
\geometry{
    textheight=9in,
    textwidth=5.5in,
    top=1in,
    headheight=12pt,
    headsep=25pt,
    footskip=30pt
}

% ------------------------------------------------------------------------------

\begin{document}

% ------------------------------------------------------------------------------
% Cover Page and ToC
% ------------------------------------------------------------------------------

\title{ \normalsize \textsc{}
		\\ [2.0cm]
		\HRule{1.5pt} \\
		\LARGE \textbf{\uppercase{Lecture 1 (May 13)}
		\HRule{2.0pt} \\ [0.6cm] \LARGE{Fields, Induction, Binomial Theorem: \textit{TEXTBOOK pg 20 - approx 16 (up to 1.2 )}} \vspace*{10\baselineskip}}
		}
\date{}
\author{\textbf{Author} \\ 
		Tom Jeong
        }

\maketitle
\newpage

\tableofcontents
\newpage

% ------------------------------------------------------------------------------

\section{Natural Numbers and Real Numbers}
Natural numbers are a set of positive integers: $$\mathbb{N} = \{1, 2, 3, 4, \dots \}$$
Real numbers are a set of all rational and irrational numbers: $$\mathbb{R} = \mathbb{Q} \cup \mathbb{I}$$
\subsection{Addition and Multiplication as a function on $\mathbb{R}$}
\begin{enumerate}
    \item Addition: $+ : \mathbb{R}^2 \rightarrow \mathbb{R}$
    \item Multiplication: $\cdot : \mathbb{R}^2 \rightarrow \mathbb{R}$
\end{enumerate}
\section{Fields}
\begin{axiom}[Field Axiom]
    A field is a set of numbers $\mathbb{F}$ with two operations: addition and multiplication. A field must satisfy the following properties:
    \begin{enumerate}
        \item Closure property under addition and multiplication: $a + b \in \mathbb{F}$ and $a \cdot b \in \mathbb{F}$
        \item Associative: $(a + b) + c = a + (b + c)$ and $(a \cdot b) \cdot c = a \cdot (b \cdot c)$
        \item Commutative: $a + b = b + a$ and $a \cdot b = b \cdot a$
        \item Distributive: $a \cdot (b + c) = a \cdot b + a \cdot c$
        \item Unique Additive Identity: There exists a unique element $0 \in \mathbb{F} \text{  such that } \forall x \in \mathbb{F}$, $0 + x = x$.
        \item Unique Multiplicative Identity: There exists a unique element $1 \in \mathbb{F} \text{  such that } \forall x \in \mathbb{F}$, $1 \cdot x = x$.
        \item Additive Inverse: For every $x \in \mathbb{F}$, there exists a unique element $-x \in \mathbb{F} \text{  such that } x + (-x) = 0$.
        \item Multiplicative Inverse: For every $x \in \mathbb{F}, x \not = 0$, there exists a unique element $x^{-1} \in \mathbb{F} \text{  such that } x \cdot x^{-1} = 1$.
    \end{enumerate}
\end{axiom}
A field is a set of numbers with two operations: addition and multiplication. A field must satisfy the following properties:
\begin{enumerate}
    \item Associative: $(a + b) + c = a + (b + c)$ and $(a \cdot b) \cdot c = a \cdot (b \cdot c)$
    \item Commutative: $a + b = b + a$ and $a \cdot b = b \cdot a$
    \item Distributive: $a \cdot (b + c) = a \cdot b + a \cdot c$
\end{enumerate}
We can see that $\mathbb{R}$ is a field since it satisfies all three properties. 

\subsection{Additive Identity}
There exists a unique element $0 \in \mathbb{R} \text{  such that } \forall x \in \mathbb{R}$, $0 + x = x$. 
\subsection{Multiplicative Identity}
There exists a unique element $1 \in \mathbb{R} \text{  such that } \forall x \in \mathbb{R}$, $1 \cdot x = x$.
\subsection{Additive Inverse}
For every $x \in \mathbb{R}$, there exists a unique element $-x \in \mathbb{R} \text{  such that } x + (-x) = 0$.

\subsection{Multiplicative Inverse}
For every $x \in \mathbb{R}$, there exists a unique element $x^{-1} \in \mathbb{R} \text{  such that } x \cdot x^{-1} = 1$.

Using these additional properties, we can prove that $(-1)^2 = 1$.
\section{Proof: $(-1)^2 = 1$}
\begin{proofbox}
Let $1_L$ be the multiplicative identity and $-1$ be the additive inverse of the multiplicative identity. We want to show that $\forall x \in \mathbb{R} : (-1) \cdot (-1) x = x$. Then by the uniqueness of the multiplicative identity, $(-1)^2$ must be the multiplicative identity. We will show:
\begin{align*}
    (-1)^2x  &= x \\
    (-1) + (1) &= 0 \\ 
    ((-1) + 1) \cdot x &= 0\cdot x \\ 
    (-1) \cdot x + 1 \cdot x &= 0 \\ 
    (-1) \cdot x+ x &= 0 \\ 
    (-1)^2 \cdot x - x &= 0 \\ 
    (-1)^2 \cdot x -x + x&= 0 + x \\ 
    (-1)^2 \cdot x &= x \\
\end{align*}
Thus we have shown that $(-1)^2$ is the multiplicative identity and thus by the uniqueness of the multiplicative identity, $(-1)^2$ must be 1.
\end{proofbox}

\section{Ordered Real Numbers}
\begin{definition}
    $$<: \mathbb{R}^2 \longrightarrow \{\textit{True},\textit{False} \} \text{ (or using binary) } \{0, 1\}$$
\end{definition}
The set of real numbers $\mathbb{R}$ is ordered by a relation $<: \mathbb{R}^2 \rightarrow \{0, 1\}$ with the following properties:
\begin{enumerate}
    \item Trichotomy: $\forall x, y \in \mathbb{R} : (x < y) \lor (x = y) \lor (y < x)$
    \item Transitivity: $\forall x, y, z \in \mathbb{R} : (x < y) \land (y < z) \rightarrow (x < z)$
    \item Additivity: $\forall x, y, z \in \mathbb{R} : (x < y) \rightarrow (x + z < y + z)$
    \item Multiplicative: $\forall x, y, z \in \mathbb{R} : (x < y) \land (0 < z) \rightarrow (x \cdot z < y \cdot z)$ and $(x < y) \land (z < 0) \rightarrow (y \cdot z < x \cdot z)$
    
\end{enumerate}

\section{Prove the Binomial Expansion Formula}
The binomial expansion formula states that $$(a + b)^n = \sum_{i = 0}^{n} \binom{n}{i}a^{n-i} b^{i}$$
\begin{axiom}
    Given any non-empty subset $E \subseteq \mathbb{N}$, there is a least element $\alpha \in E \text{ such that } \alpha \leq x \text{   } \forall x \in E$.
\end{axiom}

\begin{theorem}[Theorem of Induction]
Consider a statement dependent on $\mathbb{N}$: $$S: \mathbb{N} \longrightarrow \{0, 1\}$$
If: \\ 
\begin{enumerate}
    \item $S(1) = 1$ 
    \item $\forall n \in \mathbb{N} : S(n) \rightarrow S(n+1)$
    \item then $\forall n \in \mathbb{N} : S(n)$
\end{enumerate}
\begin{proofbox}
Consider $E = \{n \in \mathbb{N} : S(n) = 0\}$.
 Suppose $E \neq \emptyset$. 
 Then by the axiom of $\mathbb{N}$, 
 there is a least element $\alpha \in E$.
  $\alpha \neq 1$ because $S(1) = 1$. 
  Consider $\alpha - 1$. $S(\alpha - 1)$ 
  must be true because $\alpha$ is the least element of $E$.
   Because $S(\alpha - 1) \rightarrow S(\alpha)$ and $S(\alpha) = 0$.
    Thus our supposition is wrong. Therefore $E = \emptyset$ and $\forall n \in \mathbb{N} : S(n)$. Q.E.D

\end{proofbox}
\end{theorem}

\begin{lemma}[Binomial Theorem]
    $$\binom{n+1}{k} = \binom{n}{k} + \binom{n}{k-1}$$
\end{lemma}
N ow we can prove the binomial expansion formula using induction.
\subsection{Proof of the Binomial Theorem by Induction}
\begin{proofbox}
We will prove the binomial theorem using mathematical induction. \\ 
\textit{Base Case:} For $n = 0$, the binomial theorem states that $(a + b)^0 = 1$. This is true since any number raised to the power of 0 is equal to 1.

\textit{Inductive Hypothesis:} Assume that the binomial theorem holds for some positive integer $k$, i.e., $(a + b)^k = \sum_{i=0}^{k} \binom{k}{i} a^{k-i} b^i$.

\textit{Inductive Step:} We need to show that the binomial theorem holds for $k+1$, i.e., $(a + b)^{k+1} = \sum_{i=0}^{k+1} \binom{k+1}{i} a^{k+1-i} b^i$.

Expanding $(a + b)^{k+1}$ using the distributive property, we get:
$$(a + b)^{k+1} = (a + b)(a + b)^k$$

Using the inductive hypothesis, we can rewrite this as:
$$(a + b)^{k+1} = (a + b) \left(\sum_{i=0}^{k} \binom{k}{i} a^{k-i} b^i\right)$$

Expanding the product, we have:
$$(a + b)^{k+1} = \sum_{i=0}^{k} \binom{k}{i} a^{k+1-i} b^i + \sum_{i=0}^{k} \binom{k}{i} a^{k-i} b^{i+1}$$

Now, let's focus on the terms in the second sum. We can rewrite them as:
$$\sum_{i=0}^{k} \binom{k}{i} a^{k-i} b^{i+1} = \sum_{i=1}^{k+1} \binom{k}{i-1} a^{k+1-i} b^i$$

Combining the two sums, we get:
$$(a + b)^{k+1} = \sum_{i=0}^{k+1} \left(\binom{k}{i} + \binom{k}{i-1}\right) a^{k+1-i} b^i$$

Using the lemma $\binom{n+1}{k} = \binom{n}{k} + \binom{n}{k-1}$, we can simplify the expression further:
$$(a + b)^{k+1} = \sum_{i=0}^{k+1} \binom{k+1}{i} a^{k+1-i} b^i$$

This completes the inductive step.

By the principle of mathematical induction, the binomial theorem holds for all positive integers $n$.

Therefore, the binomial expansion formula $$(a + b)^n = \sum_{i = 0}^{n} \binom{n}{i}a^{n-i} b^{i}$$ is proven.
\end{proofbox}

\section{Last Property of $\mathbb{R}$}
Given any bounded non-empty subset $E \subseteq \mathbb{R}$, there exists a supremum (the least upper bound). (continued next lecture )
% Maybe I need to add one more part: Examples.
% Set style and colour later.


\end{document}
