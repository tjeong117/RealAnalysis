\documentclass{article}
\usepackage{amsmath, amsthm, amssymb, amsfonts}
\usepackage{thmtools}
\usepackage{graphicx}
\usepackage{setspace}
\usepackage{geometry}
\usepackage{float}
\usepackage{hyperref}
\usepackage[utf8]{inputenc}
\usepackage[english]{babel}
\usepackage{framed}
\usepackage[dvipsnames]{xcolor}
\usepackage{tcolorbox}

\colorlet{LightGray}{White!90!Periwinkle}
\colorlet{LightOrange}{Orange!15}
\colorlet{LightGreen}{Green!15}

\newcommand{\HRule}[1]{\rule{\linewidth}{#1}}

\colorlet{LightGray}{black!10}
\colorlet{LightOrange}{orange!15}
\colorlet{LightGreen}{green!15}
\colorlet{LightBlue}{blue!15}
\colorlet{LightCyan}{cyan!15}



\declaretheoremstyle[name=Theorem,]{thmsty}
\declaretheorem[style=thmsty,numberwithin=section]{theorem}
\usepackage{tcolorbox} % Add missing package
\tcolorboxenvironment{theorem}{colback=LightGray}

\declaretheoremstyle[name=Definition,]{thmsty}
\declaretheorem[style=thmsty,numberwithin=section]{definition}
\tcolorboxenvironment{definition}{colback=LightBlue}

\declaretheoremstyle[name=Proposition,]{prosty}
\declaretheorem[style=prosty,numberlike=theorem]{proposition}
\tcolorboxenvironment{proposition}{colback=LightOrange}


\declaretheoremstyle[name=Proof,]{prosty}
\declaretheorem[style=prosty,numberlike=theorem]{proofbox}
\tcolorboxenvironment{proofbox}{colback=LightOrange}

\declaretheoremstyle[name=Axiom,]{prcpsty}
\declaretheorem[style=prcpsty,numberlike=theorem]{axiom}
\tcolorboxenvironment{axiom}{colback=LightGreen}

\declaretheoremstyle[name=Lemma,]{prcpsty}
\declaretheorem[style=prcpsty,numberlike=theorem]{lemma}
\tcolorboxenvironment{lemma}{colback=LightCyan}





\setstretch{1.0}
\geometry{
    textheight=9in,
    textwidth=7in,
    top=1in,
    headheight=12pt,
    headsep=25pt,
    footskip=30pt
}

% ------------------------------------------------------------------------------

\begin{document}

% ------------------------------------------------------------------------------
% Cover Page and ToC
% ------------------------------------------------------------------------------

\title{ \normalsize \textsc{}
		\\ [2.0cm]
		\HRule{1.5pt} \\
		\LARGE \textbf{\uppercase{Lecture NUMBER (DATE)}
		\HRule{2.0pt} \\ [0.6cm] \LARGE{} \vspace*{10\baselineskip}}
		}
\date{}
\author{\textbf{Author} \\ 
		Tom Jeong
        }

\maketitle
\newpage

\tableofcontents
\newpage

% ------------------------------------------------------------------------------
\section{Completion}
$R \approxeq \text{Completion of } Q$ \\
w.r.t $e / \sim $ $= \{[(a_n) ] : a_n \in e \} $ where $e = \{ (a_n) : a_n \in Q \}$ and $(a_n)$ is cauchy, i.e. $\forall \epsilon > 0, \exists N \in \mathbb{N}$ s.t. $n, m \geq N \implies |a_n - a_m| < \epsilon$ \\ 
we want to show \\
1. $(a_n)$ is cauchy $\forall a \in \mathbb{R} \rightarrow a \cdot (a_n)$ is also cauchy. What does it mean to multiply a number to a sequence? it means to multiply a number to each element of the sequence. i.e. $(a \cdot a_n)$\\
2. $(a_n)$ and $(b_n)$ are cauchy $\rightarrow (a_n + b_n)$ is also cauchy. \\ 
3. if $(a_n) + (b_n)$ is cauchy, then $(a_n)(b_n)$ are cauchy. \\

\begin{proofbox}[proof of 2, 3]
    $(a_n)$ and $(b_n)$ are cauchy $\rightarrow (a_n + b_n)$ is also cauchy. \\
    let $\epsilon > 0, \exists N \text{ such that } \forall n_1, n_2, m_1, m_2 > N \implies |a_{n_1}  - a_{m_1}| < \epsilon    \text{ and } |b_{n_2} - b_{m_2}| < \epsilon$ \\

    w.t.s $\exists M \text{ such that } \forall n_3, m_3 > M \implies |(a_{n_3} + b_{n_3}) - (a_{m_3} + b_{m_3})| < \epsilon$ \\ (using triangle inequality we can pair a's and b's together) \\
    $|(a_{n_3} + b_{n_3}) - (a_{m_3} + b_{m_3})| = |(a_{n_3} - a_{m_3}) + (b_{n_3} - b_{m_3})| \leq |a_{n_3} - a_{m_3}| + |b_{n_3} - b_{m_3}| < 2\epsilon$ \\

    
    3. WTS $\exists M \text{ such that } \forall n_3, m_3 > M \implies |(a_{n_3} \cdot b_{n_3}) - (a_{m_3} \cdot b_{m_3})| < \epsilon$ \\
    $ = |a_{n_3} \cdot b_{n_3} - a_{m_3} \cdot b_{m_3}| = |a_{n_3} \cdot b_{n_3} - a_{n_3} \cdot b_{m_3} + a_{n_3} \cdot b_{m_3} - a_{m_3} \cdot b_{m_3}|$  (we add an subtract) \\
    $ = |a_{n_3} \cdot (b_{n_3} - b_{m_3}) + b_{m_3} \cdot (a_{n_3} - a_{m_3})| $      \hspace{30pt} ( ... we need a lemma) using lemma 1.2 we see that the sequence converges 

\end{proofbox}
\begin{lemma}
    if $(a_n)$ is cauchy, then the sequence $(a_n)$ is bounded. 
    $$\forall \epsilon_0 > 0, \exists N \text{ such that } \forall n, m  > N \implies |a_n - a_m| < \epsilon$$ 
    in particular,if we look at $|a_n - a_N | < \epsilon$ for $n \geq N$ $$-\epsilon_0 < a_n - a_N < \epsilon_0 \implies a_N - \epsilon < a_n < \epsilon + a_N$$ 
?? reverse inequality: $|a_n| - |a_N| < |a_n - a_N| < \epsilon_0 \implies |a_n| < |a_N| + \epsilon_0$ we see that the sequence is bounded by some number; in mathematical terms $M \text{ such that } \forall n \in \mathbb{N} \implies |a_n| < M$ we can choose $M = \max\{|a_1|, |a_2|, \dots, |a_N| + \epsilon_0\}$ 
\end{lemma}
\begin{definition}
    define $$[(b_n)][(a_n)] = [(a_n \cdot b_n)]$$ 
    $$[(a_n)]([(b_n)] + [c_n]) = [(a_n)] \cdot [(b_n + c_n)] = [(a_n \cdot b_n + a_n \cdot c_n)]$$ 

\end{definition}
\begin{theorem}
    $$(a_n) \subseteq \mathbb{R}$$
    cauchy $ \longleftrightarrow (a_n)$ converges (there is a specific number that the sequence converges to) $\exists a \text{ such that } \forall \epsilon \exists N \text{ such that } \forall n \geq N \implies |a_n - a| < \epsilon$
    
    \begin{proof}
        $(a_n)$ bounded, due to B.W Theorem. there exists a convergence subsequence.  consider $(a_{n_j})_{j \geq 1}$ then $a \text{ such that } \forall \epsilon > 0, \exists J \in \mathbb{N} \text{ such that } \forall j \geq J \implies |a_{n_j} - a| < \epsilon$     \\ 
        $\exists N, \text{ such that } \forall n, m \geq N \implies |a_n - a_m| < \epsilon$ we pick $j$ large enough such that $j \geq J$ and $n_j \geq N$ we know that $|a_{n_j} - a| < \epsilon$ and $|a_{n_{j+1}} - a| < \epsilon$ we can use the triangle inequality to show that $|a_n - a| < 2\epsilon$

    \end{proof}
\end{theorem}

\section{liminf and limsup}
they are combination of words limit  infimum and limit supremum.
\begin{definition}[Limsup]
    given $(a_n)_{n \geq 1}$ define $limsup(a_n)$ which is not the same concept as $\sup\{a_n\}$ they are not equal in general. $limsup(a_n) =  \sup \{inf\{a_{n_j} : a_{n_j} \text{ is a subsequence of } (a_n)\} \}$ 
    $(a_n)$ is bounded then = $\sup\{ \lim_{n \to \infty} \sup\{a_{n_j} \}\}$
    thus $= \lim \sup\{a_n\} = \lim_{n \to \infty} \{\sup\{a_m\}\}$ $m \geq N$ 
\end{definition}
\begin{definition}[Liminf]
    $liminf(a_n) = \inf\{ \sup\{a_{n_j} : a_{n_j} \text{ is a subsequence of } (a_n)\} \}$ 
    $(a_n)$ is bounded then = $\inf\{ \lim_{n \to \infty} \inf\{a_{n_j} \}\}$
    thus $= \lim \inf\{a_n\} = \lim_{n \to \infty} \{\inf\{a_m\}\}$ $m \geq N$
\end{definition}

\begin{definition}[Cluster Point ]
    $a$ is a cluster point of $(a_n)_{n \geq 1}$ if $\exists $ subsequence $(a_{n_j} ) _{j \geq 1} \text{ such that } \lim_{j \to \infty} a_{n_j} = a$

\end{definition}
Here is an example, make a subsequence with cluster points 1, 2, 3, and $\infty$
\begin{align*}
    a_n = \begin{cases}
        1 & \text{if } n = 1 \mod 4\\ 
        2 & \text{if } n = 2 \mod 4\\
        3 & \text{if } n = 3 \mod 4\\
        n & \text{if } n = 0 \mod 4\\
    \end{cases}
\end{align*}
then $lim\sup(a_n) = \infty$ and $lim\inf(a_n) = 1$

\begin{theorem}
    $\forall (a_n)_{n \geq 1}$ cluster points of $(a_n)$ are in the interval $[\lim\inf(a_n), \lim\sup(a_n)]$
    \begin{proof}
        Recall the definiition, $a_n = liminf(a_n) = \inf\{ \sup\{a_{n_j} : a_{n_j} \text{ is a subsequence of } (a_n)\} \}$
         and , $\Bar{a_n} = limsup(a_n) =   \sup\{ \inf\{a_{n_j} : a_{n_j} \text{ is a subsequence of } (a_n)\} \}$ this means $a_n = inf\{ \sup\{a_{n_j} : a_{n_j} \text{ is a subsequence of } (a_n)\} \leq \sup\{a_{n_j} : a_{n_j} \text{ is a subsequence of } (a_n)\} \leq \sup\{a_m\}$ for $m \geq N$ and $a_n \leq \sup\{a_m\}$ for $m \geq N$
         let $a$, be a cluster point $\implies \exists$ subsequence $(a_{n_j})_{j \geq 1}$ s.t. $\lim_{j \to \infty} a_{n_j} = a$

    \end{proof}
\end{theorem}
Corollary:     if $lim\sup(a_n) = lim\inf(a_n)$ then $(a_n)$ converges. in this case all three are equal
\end{document}
