\documentclass{article}
\usepackage{amsmath, amsthm, amssymb, amsfonts}
\usepackage{thmtools}
\usepackage{graphicx}
\usepackage{setspace}
\usepackage{geometry}
\usepackage{float}
\usepackage{hyperref}
\usepackage[utf8]{inputenc}
\usepackage[english]{babel}
\usepackage{framed}
\usepackage[dvipsnames]{xcolor}
\usepackage{tcolorbox}

\colorlet{LightGray}{White!90!Periwinkle}
\colorlet{LightOrange}{Orange!15}
\colorlet{LightGreen}{Green!15}

\newcommand{\HRule}[1]{\rule{\linewidth}{#1}}

\colorlet{LightGray}{black!10}
\colorlet{LightOrange}{orange!15}
\colorlet{LightGreen}{green!15}
\colorlet{LightBlue}{blue!15}
\colorlet{LightCyan}{cyan!15}



\declaretheoremstyle[name=Theorem,]{thmsty}
\declaretheorem[style=thmsty,numberwithin=section]{theorem}
\usepackage{tcolorbox} % Add missing package
\tcolorboxenvironment{theorem}{colback=LightGray}

\declaretheoremstyle[name=Definition,]{thmsty}
\declaretheorem[style=thmsty,numberwithin=section]{definition}
\tcolorboxenvironment{definition}{colback=LightBlue}

\declaretheoremstyle[name=Proposition,]{prosty}
\declaretheorem[style=prosty,numberlike=theorem]{proposition}
\tcolorboxenvironment{proposition}{colback=LightOrange}


\declaretheoremstyle[name=Proof,]{prosty}
\declaretheorem[style=prosty,numberlike=theorem]{proofbox}
\tcolorboxenvironment{proofbox}{colback=LightOrange}

\declaretheoremstyle[name=Axiom,]{prcpsty}
\declaretheorem[style=prcpsty,numberlike=theorem]{axiom}
\tcolorboxenvironment{axiom}{colback=LightGreen}

\declaretheoremstyle[name=Lemma,]{prcpsty}
\declaretheorem[style=prcpsty,numberlike=theorem]{lemma}
\tcolorboxenvironment{lemma}{colback=LightCyan}





\setstretch{1.2}
\geometry{
    textheight=9in,
    textwidth=5.5in,
    top=1in,
    headheight=12pt,
    headsep=25pt,
    footskip=30pt
}

% ------------------------------------------------------------------------------

\begin{document}

% ------------------------------------------------------------------------------
% Cover Page and ToC
% ------------------------------------------------------------------------------

\title{ \normalsize \textsc{}
		\\ [2.0cm]
		\HRule{1.5pt} \\
		\LARGE \textbf{\uppercase{June  10}
		\HRule{2.0pt} \\ [0.6cm] \LARGE{{set stuff}} \vspace*{10\baselineskip}}
		}
\date{}
\author{\textbf{Author} \\ 
		Tom Jeong
        }

\maketitle
\newpage

\tableofcontents
\newpage

% ------------------------------------------------------------------------------
\section{Connected Set}
A set $E$ is connected if $\not \exists$ a separation of $E$ into two open componenets. $E = V_1 \cup V_2$ and $V_1 \cap V_2 = \emptyset$ and $V_1 \neq \emptyset \neq V_2$ and $\bar{V_1} \cap V_2 = V_1 \cap \bar{V_2} = \emptyset$ and $V_1 \neq E \neq V_2$.

\begin{theorem}
    $E \subseteq \mathbb{R}$, $E$ connected is equivalent to $E$ is on interval $I$.
\end{theorem}
\begin{proof}
    suppose $E \subseteq \mathbb{R}$ is connected. solution implying $H_1 \subseteq H_2 \subseteq H_3 \dots$ (sequences in $\mathbb{R}^n$)
    E is empty or, if $E$ contains only $c$, $E = (c, c)$ or $[c,c]$. suppose $E$ contains at least 2 points. 
    $$a = \inf(E)$$
    $$b = \sup(E)$$ and $a < b$ 
    two casese we can take: $a \in E$ or $a \not in E$, $b \in E$ or $b \not in E$. property of supremums and infimums.

    
    consider $E_k = [a_k, b_k]$ Q: is $E_k \subseteq E$ ? $E$ is one of the following: $E = (a,b)$, $E = [a,b]$, $E = (a,b]$, $E = [a,b)$
    how to prove $E_k \subseteq E$? suppose not then, $\exists x \in E_k$ such that $x \not \in E$ $a < a_k \leq x \leq b_k < b$
    consider $E \setminus \{x\} = E$ $E_R = \{ y\in E : y > x\} $

    
    \begin{definition}
        $U$ is erlatively open wrt $E$ if $\exists$ open set $A$ such that $U = E \cap A$ 
    \end{definition}
    by $E = [a,b]$ and $ U = E \cap A$ then $A = (\frac{a+b}{2}, b + \epsilon)$ and $U = [\frac{a+b}{2}, b]$
 \end{proof}

 back to $E_R$, show $\exists V$ open s.t. $V \cap E = E_2$


    $\bigcup_{k = 1}^{\infty} E_k \subseteq E$ $ x \in E_k$ for some k which meanns $x \in E$
    \\ case 1: $a_k = a \in E$ 
    true
    \\ case 2: $a_k > a \not \in E:$
    $x \in \tilde{E} \rightarrow x \in E_k $ for some $k \rightarrow a < a_k \leq x$ \



    \begin{definition}
        A set $E $ is perfect if $E = $ limit pots of $E$ \\ 
        limit points of $E: \{x \in X : \forall B \text{ open, } x \in B, \exists y (\not = x) \in E \cap N\}$ 
    \end{definition}
    \begin{definition}
        A set $E$ is nowhere dense if $\bar{A}$ (closure of $A$) has empty interior 
    \end{definition}
    $\bar{E} = \bigcap V$, $\{V \text{ closed } : V \supseteq E\}$
    $E^0 = \{x \in E : \exists V \text{ open } x \in V \subseteq E\}$



    Cantor Set: 
    $C = \bigcap_{k = 1}^{\infty} I_k$ where $I_1 = [0,1]$, $I_2 = [0, \frac{1}{3}] \cup [\frac{2}{3}, 1]$, $I_3 = [0, \frac{1}{9}] \cup [\frac{2}{9}, \frac{3}{9}] \cup [\frac{6}{9}, \frac{7}{9}] \cup [\frac{8}{9}, 1]$
    Q: is C closed? yes intersection of closed $I_k$ \\ 
    Q: is C compact, yes, closed and bounded. \\ 
    Q: is  perfect? \\ 
    Q: is C nowhere dense?

    $$C = \{\sum_{k = 1}^{\infty} a_k (\frac{1}{3})^k: a_k \in \{0, 2\}\}$$
    $ x \in C, \epsilon > 0, \text{ show } B_{\epsilon} (x) 

\end{document}
