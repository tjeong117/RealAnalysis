\documentclass{article}
\usepackage{amsmath, amsthm, amssymb, amsfonts}
\usepackage{thmtools}
\usepackage{graphicx}
\usepackage{setspace}
\usepackage{geometry}
\usepackage{float}
\usepackage{hyperref}
\usepackage[utf8]{inputenc}
\usepackage[english]{babel}
\usepackage{framed}
\usepackage[dvipsnames]{xcolor}
\usepackage{tcolorbox}

\colorlet{LightGray}{White!90!Periwinkle}
\colorlet{LightOrange}{Orange!15}
\colorlet{LightGreen}{Green!15}


\colorlet{LightGray}{black!10}
\colorlet{LightOrange}{white!15}
\colorlet{LightGreen}{green!15}
\colorlet{LightBlue}{blue!15}
\colorlet{LightCyan}{cyan!15}



\declaretheoremstyle[name=Theorem,]{thmsty}
\declaretheorem[style=thmsty,numberwithin=section]{theorem}
\usepackage{tcolorbox} % Add missing package
\tcolorboxenvironment{theorem}{colback=LightGray}

\declaretheoremstyle[name=Definition,]{thmsty}
\declaretheorem[style=thmsty,numberwithin=section]{definition}
\tcolorboxenvironment{definition}{colback=LightBlue}

\declaretheoremstyle[name=Proposition,]{prosty}
\declaretheorem[style=prosty,numberlike=theorem]{proposition}
\tcolorboxenvironment{proposition}{colback=LightOrange}


\declaretheoremstyle[name=Proof,]{prosty}
\declaretheorem[style=prosty,numberlike=theorem]{proofbox}
\tcolorboxenvironment{proofbox}{colback=LightOrange}

\declaretheoremstyle[name=Axiom,]{prcpsty}
\declaretheorem[style=prcpsty,numberlike=theorem]{axiom}
\tcolorboxenvironment{axiom}{colback=LightGreen}

\declaretheoremstyle[name=Lemma,]{prcpsty}
\declaretheorem[style=prcpsty,numberlike=theorem]{lemma}
\tcolorboxenvironment{lemma}{colback=LightCyan}





% ------------------------------------------------------------------------------

\begin{document}

% ------------------------------------------------------------------------------
% Cover Page and ToC
% ------------------------------------------------------------------------------

\title{Homework 2 Tom (wonsuk) Jeong  }
\maketitle

\section{Question 1} Let $a_1 = 1$ and $a_{n+1} = 1 + \frac{1 + a_n}{1 + a_n}$. Find $\lim_{n \to \infty} a_n$.
 hint: Observe that $a_{n+1} = 1 + \frac{2a_n}{1 + a_n}$. Then write $a_{n+k} = \cfrac{A_k + B_k a_n}{C_k + D_k a_n}$  and find a formula for $A_n, B_n, C_n, D_n$. Then observe you can write $a_{n+1} = \cfrac{A_n + B_n}{C_n + D_n}$.
    finding the $\lim a_n$ 
    \begin{align*}
        a_{n+1} &= 1 + \frac{2a_n}{1 + a_n} \\
        a_{n+2} &= 1 + \frac{2a_{n+1}}{1 + a_{n+1}} \\
        &= 1 + \frac{2(1 + \frac{2a_n}{1 + a_n})}{1 + 1 + \frac{2a_n}{1 + a_n}} \\
        &= 1 + \frac{2 + \frac{4a_n}{1 + a_n}}{2 + \frac{2a_n}{1 + a_n}} \\
        &= 1 + \frac{2 + 4a_n}{2 + 2a_n} \\
        a_{n+3} &= 1 + \frac{2 + 4a_{n+1}}{2 + 2a_{n+1}} \\
        &= 1 + \frac{2 + 4(1 + \frac{2a_n}{1 + a_n})}{2 + 2(1 + \frac{2a_n}{1 + a_n})} \\
        &= 1 + \frac{2 + 4 + \frac{8a_n}{1 + a_n}}{2 + 2 + \frac{4a_n}{1 + a_n}} \\
        &= 1 + \frac{6 + 8a_n}{4 + 4a_n} \\
    \end{align*}
    generalizing the pattern we get 
    \begin{align*}
        a_{n+k} &= 1 + \frac{2k + 2a_n}{k + a_n} \\
    \end{align*}
    we can write $a_{n+k} = \cfrac{A_k + B_k a_n}{C_k + D_k a_n}$ by setting $A_k = 2k + 2, B_k = 1, C_k = k, D_k = 1$.
    \begin{align*}
        a_{n+k} &= \cfrac{2k + 2 + a_n}{k + a_n} \\
    \end{align*}
    then $a_{n + 1} = \cfrac{A_n + B_n}{C_n + D_n}$ by setting $A_n = 2 + 2, B_n = 1, C_n = 1, D_n = 1$.
    \begin{align*}
        a_{n+1} &= \cfrac{4 + a_n}{1 + a_n} \\
    \end{align*}
    to find the limit of $a_n$ we can set $a_{n+1} = a_n = a$ and solve for $a$.
    \begin{align*}
        a &= \cfrac{4 + a}{1 + a} \\
        a + a^2 &= 4 + a \\
        a^2 &= 4 \\
        a &= 2, -2 \\
    \end{align*}
    since $a_1 = 1$ we have $a = 2$.
    \begin{align*}
        \lim_{n \to \infty} a_n = 2 \\
    \end{align*}

\section{question 2}


Let $0 < x_1 \leq x_2 \leq \dots \leq x_k$ prove that $\lim_{n \to \infty} (x_{1}^n + x_{2}^n + \dots + x_{k}^n)^{1/n} = x_k$.
\begin{proofbox}
    Consider the sequence \(\{a_n\}\) defined by
    \[
    a_n = (x_1^n + x_2^n + \cdots + x_k^n)^{1/n}.
    \]
    We want to show that
    \[
    \lim_{n \to \infty} a_n = x_k.
    \]
    
    First, observe that since \(0 < x_1 \leq x_2 \leq \cdots \leq x_k\), we have:
    \[
    x_k^n \leq x_1^n + x_2^n + \cdots + x_k^n \leq kx_k^n.
    \]
    
    Taking the \(n\)-th root of each part of the inequality, we get:
    \[
    (x_k^n)^{1/n} \leq (x_1^n + x_2^n + \cdots + x_k^n)^{1/n} \leq (kx_k^n)^{1/n}.
    \]
    
    Simplifying:
    \[
    x_k \leq (x_1^n + x_2^n + \cdots + x_k^n)^{1/n} \leq k^{1/n} \cdot x_k.
    \]
        
    lhs, $\lim_{n \to \infty} x_k = x_k$ and rhs, $\lim_{n \to \infty} k^{1/n} \cdot x_k = x_k$. Thus, by squeeze theorem the limit of \(a_n\) is \(x_k\).


\end{proofbox}

\section{question 3}

Let $k \in \mathbb{N}$ and suppose $\sum_{n=1}^{\infty} |a_{n+k} - a_n|$ converges. Prove that $(a_n)$ has a convergent subsequence.Hint: Use Bolzano Weierstrass,
and that the tail end of a convergent series must go to 0

given that $\sum_{n=1}^{\infty} |a_{n+k} - a_n|$ converges, we can wrie te partial sums by $S_N = \sum_{n=1}^{N} |a_{n+k} - a_n|$. Since this series converges and $(S_N)$ is bounded\\ there exists some $M > 0$ s.t.:

\[
S_N \leq M \quad \forall N.
\]

since the series $\sum_{n=1}^{\infty} |a_{n+k} - a_n|$ converges, $|a_{n+k} - a_n| \to 0$

\[
\lim_{n \to \infty} |a_{n+k} - a_n| = 0.
\]

The BW states that every bounded sequence in $\mathbb{R}$ has a convergent subsequence. To apply this we show that $(a_n)$ is bounded.

Since $\sum_{n=1}^{\infty} |a_{n+k} - a_n|$ converges, and we know $\lim_{n \to \infty} |a_{n+k} - a_n| = 0$, we have:

\[
|a_{n+k} - a_n| < \epsilon \quad \text{for large large } n.
\]

This means that for large $n$, $a_{n+k}$ becomes closer too $a_n$. we have  

1. The first $k$ terms $a_1, a_2, \ldots, a_k$ which form a finite set and bounded.
2. For $n > k$, each $a_n$ is close to some $a_{n-k}$ due to the convergence to 0 of the differences.

We can cover the sequence by finite bounds up to $a_k$ and then note that subsequent terms stay close to these, which makes the entire sequence bounded.

Given that $(a_n)$ is bounded, by BW, there exists a convergent subsequence $(a_{n_j})$ of $(a_n)$.

thus, the sequence $(a_n)$ has a convergent subsequence.

\hfill $\blacksquare$

\section{question 4}
Prove that a sequence in a finite set has a constant subsequence.

Let $S$ be a finite set. Then $S = \{a_1, a_2, \ldots, a_n\}$ for some $n \in \mathbb{N}$. Consider a sequence $(a_n)$ in $S$. Since $S$ is finite, the sequence $(a_n)$ will repeat. so there exists some $N \in \mathbb{N}$ st $a_n = a_{n+k}$  $\forall n \geq N$ and some $k \in \mathbb{N}$.

\section{question 5} let $(n_k)$ be a sequence of positive integers which contains every
positive integer exactly once. prove that $\lim_{k} \sup a_{n_k} = \lim_{n} \sup a_n$ and $\lim_{k} \inf a_{n_k} = \lim_{n} \inf a_n$.

    let $(n_k)$ be sequence of positive integers that contains every positive integer once.we show that
    1. \[
    \lim_{k} \sup a_{n_k} = \lim_{n} \sup a_n
    \]
   2.  \[
    \lim_{k} \inf a_{n_k} = \lim_{n} \inf a_n.
    \]
    
    so, we show that
    \[
    \lim_{k} \sup a_{n_k} \leq \lim_{n} \sup a_n.
    \]
    
    Let \(M = \lim_{n} \sup a_n\). Then, $\forall$ \(\epsilon > 0\), $\exists$ \(N \in \mathbb{N}\) such that
    \[
    a_n < M + \epsilon \quad \forall n \geq N.
    \]
    
    Since \((n_k)\) contains every positive integer exactly once, there exists \(K \in \mathbb{N}\) such that
    \[
    n_k \geq N \quad \forall k \geq K.
    \]
    
    thus,  \(\forall k \geq K\), we have
    \[
    a_{n_k} < M + \epsilon.
    \]
    
    Taking the supremum over all \(k \geq K\), we get
    \[
    \sup_{k \geq K} a_{n_k} \leq M + \epsilon.
    \]
    
    \[
    \lim_{k} \sup a_{n_k} \leq \lim_{n} \sup a_n.
    \]
    we show that
    \[
    \lim_{k} \sup a_{n_k} \geq \lim_{n} \sup a_n.
    \]
    
    Let \(M = \lim_{n} \sup a_n\). $\forall$ \(\epsilon > 0\), by the definition of \(\limsup\), there are infinitely many \(n\) such that
    \[
    a_n > M - \epsilon.
    \]
    
    Since \((n_k)\) contains every positive integer exactly once, there are infinitely many \(k\) such that
    \[
    a_{n_k} > M - \epsilon.
    \]
    
    Taking the supremum the subsequence, 
    \[
    \sup a_{n_k} \geq M - \epsilon.
    \]
    
    Since \(\epsilon > 0\) we have
    \[
    \lim_{k} \sup a_{n_k} \geq \lim_{n} \sup a_n.
    \]
    
    Combining the two inequalities, we have
    \[
    \lim_{k} \sup a_{n_k} = \lim_{n} \sup a_n.
    \]
    
    Now, we show that
    \[
    \lim_{k} \inf a_{n_k} \leq \lim_{n} \inf a_n.
    \]
    
    Let \(m = \lim_{n} \inf a_n\). Then  \(\forall \epsilon > 0\), there exists \(N \in \mathbb{N}\) such that
    \[
    a_n > m - \epsilon \quad \forall n \geq N.
    \]
    
    Since \((n_k)\) contains every positive integer exactly once, there exists \(K \in \mathbb{N}\) such that
    \[
    n_k \geq N \quad \forall k \geq K.
    \]
    
    Therefore, \(\forall k \geq K\), we have
    \[
    a_{n_k} > m - \epsilon.
    \]
    
    Taking the infimum over all \(k \geq K\), we get
    \[
    \inf_{k \geq K} a_{n_k} \geq m - \epsilon.
    \]
    
    Since \(\epsilon > 0\), we have
    \[
    \lim_{k} \inf a_{n_k} \leq \lim_{n} \inf a_n.
    \]
    
    now we show that
    \[
    \lim_{k} \inf a_{n_k} \geq \lim_{n} \inf a_n.
    \]
    
    Let \(m = \lim_{n} \inf a_n\). And \(\forall \epsilon > 0\), by definition of \(\liminf\), there are infinitely many \(n\) such that
    \[
    a_n < m + \epsilon.
    \]
    
    Since \((n_k)\) contains every positive integer exactly once, there are infinitely many \(k\) st
    \[
    a_{n_k} < m + \epsilon.
    \]
    
    Taking the infimum over these values, we get
    \[
    \inf a_{n_k} \leq m + \epsilon.
    \]
    
    Since \(\epsilon > 0\) we have
    \[
    \lim_{k} \inf a_{n_k} \leq \lim_{n} \inf a_n.
    \]
    
    Combining the two inequalities, we get
    \[
    \lim_{k} \inf a_{n_k} = \lim_{n} \inf a_n.
    \]
    
    thus, we have proved that
    \[
    \lim_{k} \sup a_{n_k} = \lim_{n} \sup a_n
    \]
    and
    \[
    \lim_{k} \inf a_{n_k} = \lim_{n} \inf a_n.
    \]
    


\end{document}
