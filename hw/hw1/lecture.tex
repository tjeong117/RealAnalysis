\documentclass{article}
\usepackage{amsmath, amsthm, amssymb, amsfonts}
\usepackage{thmtools}
\usepackage{graphicx}
\usepackage{setspace}
\usepackage{geometry}
\usepackage{float}
\usepackage{hyperref}
\usepackage[utf8]{inputenc}
\usepackage[english]{babel}
\usepackage{framed}
\usepackage[dvipsnames]{xcolor}
\usepackage{tcolorbox}

\colorlet{LightGray}{White!90!Periwinkle}
\colorlet{LightOrange}{Orange!15}
\colorlet{LightGreen}{Green!15}


\colorlet{LightGray}{black!10}
\colorlet{LightOrange}{white!15}
\colorlet{LightGreen}{green!15}
\colorlet{LightBlue}{blue!15}
\colorlet{LightCyan}{cyan!15}



\declaretheoremstyle[name=Theorem,]{thmsty}
\declaretheorem[style=thmsty,numberwithin=section]{theorem}
\usepackage{tcolorbox} % Add missing package
\tcolorboxenvironment{theorem}{colback=LightGray}

\declaretheoremstyle[name=Definition,]{thmsty}
\declaretheorem[style=thmsty,numberwithin=section]{definition}
\tcolorboxenvironment{definition}{colback=LightBlue}

\declaretheoremstyle[name=Proposition,]{prosty}
\declaretheorem[style=prosty,numberlike=theorem]{proposition}
\tcolorboxenvironment{proposition}{colback=LightOrange}


\declaretheoremstyle[name=Proof,]{prosty}
\declaretheorem[style=prosty,numberlike=theorem]{proofbox}
\tcolorboxenvironment{proofbox}{colback=LightOrange}

\declaretheoremstyle[name=Axiom,]{prcpsty}
\declaretheorem[style=prcpsty,numberlike=theorem]{axiom}
\tcolorboxenvironment{axiom}{colback=LightGreen}

\declaretheoremstyle[name=Lemma,]{prcpsty}
\declaretheorem[style=prcpsty,numberlike=theorem]{lemma}
\tcolorboxenvironment{lemma}{colback=LightCyan}





% ------------------------------------------------------------------------------

\begin{document}

% ------------------------------------------------------------------------------
% Cover Page and ToC
% ------------------------------------------------------------------------------

\title{Homework 1 Tom (wonsuk) Jeong  }
\maketitle
\section{Question 1} Prove the following statement:  For all $x \in R$, $0 \cdot x = 0$, where $0$ is the additive identity of $R$. (Hint: consider $1 + 0$.)
\begin{proofbox}
    $\mathbb{R}$ is a field, so it follows the distributive property. We also know $1 + 0 = 1$ since 0 is the additive identity: 
    \begin{align*}
        1 + 0 &= 1 \\ 
        x(1 + 0) &= x(1) \text{ .. using the distributive property of fields} \\ 
        1 \cdot x + 0 \cdot x = 1 \cdot x \\
        \intertext{1 is the multiplicative identity,} 
        x + 0 \cdot x &= x \\ 
        x - x + 0\cdot x &= x - x \text{ .. $-x$ is the additive inverse of $x$ } \\ 
        0 \cdot x &= 0 \\
    \end{align*}
    We show that $0 \cdot x = 0$ for all $x \in \mathbb{R}$.
\end{proofbox}

\section{Question 2}
Write a clear proof of the Archimedian Principle, i.d. the statement: if $a, b \in \mathbb{R}_{>0}$, then there exists $n \in \mathbb{N}$ such that $a \cdot n > b$. You may use the following lemma without proving it: If $E \subset \mathbb{N}$ has a supremum $s$, then $s \in E$.
hint : consider the set $E = \{n \in \mathbb{N} : n < \frac{b}{a}\}$. Then use the lemma above and consider $a \cdot (s + 1)$.
\begin{proofbox}
    To prove: If \(a, b \in \mathbb{R}_{>0}\), then there exists \(n \in \mathbb{N}\) such that \(a \cdot n > b\).

Consider the equivalent statement: If \(a, b \in \mathbb{R}_{>0}\), then there exists \(n \in \mathbb{N}\) such that \(n > \frac{b}{a}\).

We will consider two cases:

\begin{enumerate}
    \item \textbf{Case 1: \(a \geq b\)}

    Let \(n = 1\). Then,
    \[
    a \cdot n = a \geq b.
    \]
    Thus, \(a \cdot n > b\) is satisfied for \(n = 1\).

    \item \textbf{Case 2: \(a < b\)}

    Let \(E = \{n \in \mathbb{N} : n < \frac{b}{a}\}\). 

    We know that \(E\) is non-empty since \(1 \in E\) (because \(a < b\) implies \(\frac{b}{a} > 1\)). 

    We also know that \(E\) is bounded above by \(\frac{b}{a}\). Thus, \(E \subset \mathbb{R}\) and it has a supremum \(s = \sup(E)\). By the given lemma, \(s \in E\).

    By definition of the supremum, for all \(e \in E\), we have \(e \leq s\). Also note that $s \in E \rightarrow s \in \mathbb{N}$

    Now, consider \(k = s + 1\). Since \(s\) is the supremum of \(E\), \(k \notin E\). This implies that \(k > \frac{b}{a}\). Therefore,
    \[
    a \cdot k > b.
    \]

    Since \(k = s + 1 \in \mathbb{N} \) (Natural numbers closed under addition), we have shown that there exists \(n \in \mathbb{N}\) such that \(a \cdot n > b\).
\end{enumerate}
Thus, the Archimedean Principle is proved.
    
\end{proofbox}

\section{Question 3}
Prove that $$\sum_{k = 1}^{n}k^3 = \left(\frac{n(n + 1)}{2}\right)^2$$ using mathematical induction.
\begin{proofbox}
    We will prove the statement by induction on $n$.
    \begin{enumerate}
        \item \textbf{Base Case:} $n = 1$ \\ 
        \begin{align*}
            \text{LHS: } \sum_{k = 1}^{1}k^3 &= 1^3 = 1 \\ 
            \text{RHS: } \left(\frac{1(1 + 1)}{2}\right)^2 &= 1^2 = 1 \\
        \end{align*} We get $ 1 = 1$ so the base case holds.
        \item \textbf{Inductive Hypothesis:} Assume that the statement holds for $n = m$. That is, $$\sum_{k = 1}^{m}k^3 = \left(\frac{m(m + 1)}{2}\right)^2$$
        \item \textbf{Inductive Step:} We will show that the statement holds for $n = m + 1$.
        \begin{align*}
            \sum_{k = 1}^{m + 1}k^3 &= \sum_{k = 1}^{m}k^3 + (m + 1)^3 \\ 
            &= \left(\frac{m(m + 1)}{2}\right)^2 + (m + 1)^3 \\ 
            &= \frac{m^2(m + 1)^2}{4} + (m + 1)^3 \\ 
            &= \frac{m^2(m + 1)^2 + 4(m + 1)^3}{4} \\ 
            &= \frac{(m + 1)^2(m^2 + 4m + 4)}{4} \\ 
            &= \frac{(m + 1)^2(m + 2)^2}{4} \\ 
            &= \left(\frac{(m + 1)(m + 2)}{2}\right)^2 \\
        \end{align*}
    \end{enumerate}
    Thus, by induction, we have shown that $$\sum_{k = 1}^{n}k^3 = \left(\frac{n(n + 1)}{2}\right)^2$$
\end{proofbox}
\section{Question 4}
Prove that for any pair of integers $m, n \in \mathbb{Z}$ with $n \geq 1$, there exists a pair of unique integers $q, r \in \mathbb{Z}$ such that $m = qn + r$ and $0 \leq r \leq n -1$.
\begin{proofbox}
    I will prove two things: first, the existence of $q$ and $r$ and second, the uniqueness of $q$ and $r$.
    \begin{enumerate}
        \item \textbf{Existence:} Let $m, n \in \mathbb{Z}$ with $n \geq 1 \text{ which means } n \in \mathbb{N}$. We will show that there exists $q, r \in \mathbb{Z}$ such that $m = qn + r$ and $0 \leq r \leq n - 1$.
        \\ Consider the set $S = \{m + an: a \in \mathbb{Z}\} \cap \mathbb{Z}_{\geq 0}$. Since $n \in \mathbb{N}$, $S$ is non-empty. By the Well-Ordering Principle, $S$ has a least element $r$. if $m \geq 0$ then we can take $a$ to be 1. we see that $n>0$ and $m + an = m+n \geq 0$ thus $m+n \in S$. If $m < 0$ then we can take $a$ to be $-m$. we see that $m + an = m (1-n) \geq 0$ thus $m(1-n) \in S$. \\ 
        \\ Using the well ordring principle, we see that $S$ has a least element $r$. and $r \in S$ implies $\exists a \in \mathbb{Z} \text{ such that } r = m + an$. We can write $m = qn + r$ where $q = a$ and $r = m + an$.
        \\ \\ since $r \in S$, we know that $r \geq 0$. $S \in \mathbb{Z}_{\geq 0}$ \\ \\ 
        Now we will show that $r \leq n - 1$. But since $n \in \mathbb{N}$ we can re-write this inequality as $r < n$ which is easier for this part. For contradiction, assume $r \geq n$ then we see that $r - n \in S$ because $r - n = m - nq - n = m - n(q + 1) \geq 0$. $r- n \in S \rightarrow r - n \geq r$ which is a contradiction. Thus, $r < n$ and if we write this in the terms of the question $r \leq n -1$.


        \item \textbf{Uniqueness:} We will show injection between $q, r$ to $m$. Assume that there exists $q_1, q_2, r_1, r_2 \in \mathbb{Z}$ such that $m = q_1n + r_1 = q_2n + r_2$ and $0 \leq r_1, r_2 \leq n - 1$.
        \\ \\ 
        We can write $q_1n + r_1 = q_2n + r_2$ as $q_1n - q_2n = r_2 - r_1$. This implies that $(q_1 - q_2)n = r_2 - r_1$. Since $0 \leq r_1, r_2 \leq n - 1$, we have $-n < r_2 - r_1 < n$. This implies that $-1 < q_1 - q_2 < 1$. Since $q_1, q_2 \in \mathbb{Z}$, we have $q_1 - q_2 = 0 \rightarrow q_1 = q_2$. Therefore we have $q_1 n = q_2 n \rightarrow r_1 = r_2$. Thus, $q$ and $r$ are unique.

    \end{enumerate}
    Thus, we have shown that for any pair of integers $m, n \in \mathbb{Z}$ with $n \geq 1$, there exists a pair of unique integers $q, r \in \mathbb{Z}$ such that $m = qn + r$ and $0 \leq r \leq n -1$.
\end{proofbox}


\section{Question 5}
Suppose $(a_n)$ is a sequence in $\mathbb{R}$. Prove that $(a_n)$ converges to a number $a \in \mathbb{R}$ if and only if every subsequence of $(a_n)$ converges to $a$.
\begin{proofbox}
    We will prove the statement by proving both directions.
    \begin{enumerate}
        \item \textbf{If:} Suppose $(a_n)$ converges to a number $a \in \mathbb{R}$. We will show that every subsequence of $(a_n)$ converges to $a$.
        \\ \\ 
        Let $(a_{n_k})$ be a subsequence of $(a_n)$. Since $(a_n)$ converges to $a$, we have $\lim_{n \to \infty} a_n = a$. This implies that for all $\epsilon > 0$, there exists $N \in \mathbb{N}$ such that $|a_n - a| < \epsilon$ for all $n \geq N$.
        \\ \\ 
        Since $(a_{n_k})$ is a subsequence of $(a_n)$, we have $n_k \geq k$ for all $k \in \mathbb{N}$. Thus, for all $k \geq N$, we have $n_k \geq k \geq N$. This implies that $|a_{n_k} - a| < \epsilon$ for all $k \geq N$. Therefore, $(a_{n_k})$ converges to $a$.
        \item \textbf{Only If:} Suppose every subsequence of $(a_n)$ converges to $a$. We will show that $(a_n)$ converges to $a$.
        \\ \\ 
        For contradiction, assume that $(a_n)$ does not converge to $a$. This implies that there exists $\epsilon > 0$ such that for all $N \in \mathbb{N}$, there exists $n \geq N$ such that $|a_n - a| \geq \epsilon$.
        \\ \\ 
        Consider the subsequence $(a_{n_k})$ defined as follows: $n_1 = 1$ and $n_{k + 1} > n_k$ such that $|a_{n_k} - a| \geq \epsilon$. Since $(a_{n_k})$ is a subsequence of $(a_n)$, we have $n_k \geq k$ for all $k \in \mathbb{N}$. This implies that
    
        $|a_{n_k} - a| \geq \epsilon$ for all $k \in \mathbb{N}$. Therefore, $(a_{n_k})$ does not converge to $a$, which is a contradiction.
    \end{enumerate}
    Thus, we have shown that $(a_n)$ converges to a number $a \in \mathbb{R}$ if and only if every subsequence of $(a_n)$ converges to $a$.
\end{proofbox}
\end{document}
