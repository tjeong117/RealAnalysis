\documentclass{article}
\usepackage{amsmath, amsthm, amssymb, amsfonts}
\usepackage{thmtools}
\usepackage{graphicx}
\usepackage{setspace}
\usepackage{geometry}
\usepackage{float}
\usepackage{hyperref}
\usepackage[utf8]{inputenc}
\usepackage[english]{babel}
\usepackage{framed}
\usepackage[dvipsnames]{xcolor}
\usepackage{tcolorbox}

\colorlet{LightGray}{White!90!Periwinkle}
\colorlet{LightOrange}{Orange!15}
\colorlet{LightGreen}{Green!15}

\newcommand{\HRule}[1]{\rule{\linewidth}{#1}}

\colorlet{LightGray}{black!10}
\colorlet{LightOrange}{orange!15}
\colorlet{LightGreen}{green!15}
\colorlet{LightBlue}{blue!15}
\colorlet{LightCyan}{cyan!15}



\declaretheoremstyle[name=Theorem,]{thmsty}
\declaretheorem[style=thmsty,numberwithin=section]{theorem}
\usepackage{tcolorbox} % Add missing package
\tcolorboxenvironment{theorem}{colback=LightGray}

\declaretheoremstyle[name=Definition,]{thmsty}
\declaretheorem[style=thmsty,numberwithin=section]{definition}
\tcolorboxenvironment{definition}{colback=LightBlue}

\declaretheoremstyle[name=Proposition,]{prosty}
\declaretheorem[style=prosty,numberlike=theorem]{proposition}
\tcolorboxenvironment{proposition}{colback=LightOrange}


\declaretheoremstyle[name=Proof,]{prosty}
\declaretheorem[style=prosty,numberlike=theorem]{proofbox}
\tcolorboxenvironment{proofbox}{colback=LightOrange}

\declaretheoremstyle[name=Axiom,]{prcpsty}
\declaretheorem[style=prcpsty,numberlike=theorem]{axiom}
\tcolorboxenvironment{axiom}{colback=LightGreen}

\declaretheoremstyle[name=Lemma,]{prcpsty}
\declaretheorem[style=prcpsty,numberlike=theorem]{lemma}
\tcolorboxenvironment{lemma}{colback=LightCyan}





\setstretch{1.2}
\geometry{
    textheight=9in,
    textwidth=5.5in,
    top=1in,
    headheight=12pt,
    headsep=25pt,
    footskip=30pt
}

% ------------------------------------------------------------------------------

\begin{document}

% ------------------------------------------------------------------------------
% Cover Page and ToC
% ------------------------------------------------------------------------------

\title{ \normalsize \textsc{}
		\\ [2.0cm]
		\HRule{1.5pt} \\
		\LARGE \textbf{\uppercase{June 20}
		\HRule{2.0pt} \\ [0.6cm] \LARGE{Lecture Notes}}
		}
\date{}
\author{\textbf{Author} \\ 
		Tom Jeong
        }

\maketitle
\newpage

\tableofcontents
\newpage

% ------------------------------------------------------------------------------

\section{Power series Fourier series }
\begin{definition}
    A power series is a series of the form
    \begin{equation}
        \sum_{n=0}^{\infty} a_n x^n
    \end{equation}
    where $a_n$ are constants and $x$ is a variable.
\end{definition}
\begin{definition}
    Define $R\ in \mathbb{R} \cup \{\pm \infty\}$ the radious of convergence of $S(x) = \sum_{k = 0}^{\infty}a_k (x-x_0)^k$ 
    if $S(x)$ convergent abs for $|x-x_0| < R$ and divergence $|x-x_0| > R$

\end{definition}
How to find R? \\

1.
    $R = \cfrac{1}{\limsup_{k \to \infty} |a_k|^{1/k}}$\\

    $|\sum_{k=0}^{\infty} a_k(x-x_0)^{k} | \leq \sum_{k=0}^{\infty} R^k \leq \sum_{k=0}^{\infty} |a_k(x-x_0)^k| \leq \sum_{k=0}^{\infty}  |a_k| R^k$ this is how we get the equation for R 

    2. if $R = \lim_{k \to \infty} \cfrac{|a_{k}|}{|a_{k+1}|} $ exists. then R is the radious of convergence of $S(x)$ \\

Infomrations we get from the radious of convergence :
\begin{enumerate}
    \item if $S(x)$ has radious of Convergence $R>0$ what do we know about $S(x) $ on $B_R(x_0)$? 
\end{enumerate}


\section{smoothness of functions 7.4}
continuous functions on $\mathbb{R}$ are labeled as $C^0(\mathbb{R})$ functions. (0 means no deriavtive )\\
$C^1(\mathbb{R})$ functions have continuous and the derivative is continuous \\ 
$C^2(\mathbb{R})$ functions have continuous and the first and second derivative is continuous \\
$C^{\infty}(\mathbb{R})$ all derivative are conttinuous\\ 

When does the taylor series converge? What means to be analytic 
$C^{\mu}$ - analytic: taylor series convergence and ROC is $\infty$\\ 
example: Consider 
$f(x) = \begin{cases}
    e^{-1/x^2} & x \neq 0 \\
    0 & x = 0
\end{cases}$ \\ 

f ix $C^{\infty}$ but not analytic on any interval containing 0. Proof: look up but also can take c infinity outside of 0. maybe induction 
\begin{proof}
    we want to check if f continuous at 0. 
    $$\lim_{x \to 0}e^{-\frac{1}{x^2}} = 0$$ 
    $\forall \epsilon \exists N \text{ such that } e^{-N} < \epsilon$\\ take $x << 1$ 

    out iside of x = 0: 
    $$f'(x) = \cfrac{2}{x^3}e^{-\cfrac{1}{x^2}}$$
    $$f'(0) = \lim_{h \to 0} \cfrac{f(h) - f(0)}{h}$$  lhosptial ?
\end{proof}


\begin{lemma}
    $lim_{h \to 0} \cfrac{e^{-1/h^2}}{P_n(h)} = 0$ (    $P_n(h) = $ polynomial of degree N)
    $$lim_{N to \infty} \cfrac{1}{P_n(1/N)e^{N^2}} = 0$$ want to show 
    $lim_{N\to \infty} e^{N^2}P_n(1/N) = \infty$
    $$e^{N^2}P_n(1/N) = e^{N^2} \sum_{k=0}^{n} a_k \cfrac{1}{N^k}$$
\end{lemma}


\begin{theorem}
    taylor w/ remainder
    $$P_n(x) = \sum_{k=0}^{n} \cfrac{f^{(k)}(0)}{k!}x^k$$ 
    $f(x) = P_n(x) + R_n(x)$ where $R_n(x) = \cfrac{f^{(n+1)}(c)}{(n+1)!}x^{n+1}$ for some $c \in (0,x)$
\end{theorem}

\begin{definition}[Trig series]
    A trig. series is a series like 
    $$T(x) = \cfrac{a_0}{2} + \sum_{k=1}^{\infty} a_k \cos(kx) + b_k \sin(kx)$$
    $$T_n(x) = \cfrac{a_0}{2} + \sum_{k=1}^{n} a_k \cos(kx) + b_k \sin(kx)$$
\end{definition}



\end{document}
